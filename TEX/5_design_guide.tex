\section{Design Guide} \label{designguide}
We show a detailed Step-by-step Design Guide of PCC in high frequency power converters. 
A.	Architecture
1.	Fixed Frequency 
a)	s-domain design
b)	ramp compensation design
2.	Variable frequency 
a)	digital control using 5S model

b)	analog control using describing function (ref)

\begin{align}
    
\end{align}

\subsection{Parameter selections and design equations}
1.	Shunt resistor sensor is well-suited to high frequency PCC
a)	Shunt resistor sensor is often used because of its simplicity and high bandwidth response
(1)	Ground-Reference shunt resistor sensor eases the measurement and improves the accuracy
(a)	The inductor current sensing method by a series shunt resistor is problematic in the high frequency PCC converters because of large common interference (Virginia Li, Fred Lee, 201, APEC) (Adán, TIE 2008) (Prodic 2011)
(i)	The current mirror in (Prodic 2011) is only suitable for IC.
(ii)	It is non-trivial to select a high bandwidth differential-voltage amplifier and it takes much efforts to design a circuit with good common mode noise rejection. (Adán, TIE 2008)
(b)	Although the current information is only available when the sw is turned on in ground-reference shunt resistor sensor, it is quite enough for the PCC

b)	Using inductor DCR or MOSFET Rdson (Fujio 2016 TPEL) (Adán, TIE 2008) method lacks the accuracy
(1)	 The L’s value is not accurate because of the varying current during transient. 
(2)	Both the DCR and Rdson value is changing the time
(3)	cannot work if we need the inductor to be saturated (Prodic, APEC 2012)
c)	Hall Effect sensor has limited bandwidth, (Yen-Shin Lai, 2009), 200KHz switching frequency and costly
d)	Current transformer also suffers from the limited bandwidth, in addition, it cannot measure DC current
2.	Improve the measurement bandwidth using shunt resistor
a)	RC compensation for parasitic inductor
 
 
 